\section{Consumption preference}

(LÆS MERE s. X49 i originale papir)

As in the original text we formulate the consumption preference $n_{e,t}$ the following way:

\begin{align}
    z(j, k ) &= (j + 0.7 k)^{0.75} \\
    n_{e, t} &= z(j_{e,t}, k_{e,t}) / z(2, 1)
\end{align}

where $j$ denotes the number of parents and $k$ denotes the number of kids. The notation $j_{e,t}$ denotes the average number of parents in a household conditioning on the time and education level. The same goes for $k_{e,t}$ which denotes the average number of kids in a household for a given age of agent and given education level.

In the original paper they use the PSID calculate $j_{e,t}, k_{e,t}$, and the results are not explicitely mentioned in the paper. We therefore simulate results that seem reasonable, where higher education level implies lower fertility. We find that $n_{e,t}$ is a humpshaped function with respect to $t$.
