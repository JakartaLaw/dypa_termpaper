\section{Solving the model}

To solve this model we need to solve the following problem: find the function $\varphi$ that satisfies:

\begin{equation}
    \varphi : (\mu_t, f_t, a_t , e) \mapsto (c_t^{*}, t^{*}_t, \kappa_t^{*}), V^{*}(\cdot)
\end{equation}

In other words, for a given state space which action set/choice set should the agent take, if the agent wishes to maximize its cummulative utility during its lifecycle.

The bellman equation, is what allows to only consider two consequtive periods: $(t, t+1)$. We can solve the model, by backwards induction. Solve the model in period $T$, where the model terminates.

\subsubsection{Case $\kappa_{t}=0.55$ and $i_{t}=1$}

In this case, we have
\[
\tilde{R}_{\kappa}(f_{t+1})=(1-0.55)\bar{R}+0.55\times\exp(\bar{r}+r(f_{t})+\sigma_{\epsilon}\epsilon_{t})
\]
and thus
\[
m_{t+1}=\tilde{R}_{\kappa}(f_{t+1})\left[m_{t}-c_{t}-\pi(1)-c_{d}\right]+Y_{t+1}+tr_{t+1}
\]

where all stochastic elements can be integreated out via Gauss-Hermite
integration. Thus, given an education level $e$, for each point in
the state space $(m_{t},p_{t},f_{t})$ we evaluate $4$ discrete choices
and optimize consumption given these choices. Then, we save this solution
$(c_{t}^{*},i_{t}^{*},\kappa_{t}^{*})$.

\subsection{Evaluating $E_{t}[V_{t+1}(m_{t+1},p_{t+1},f_{t+1})]$}

We are interested in evaluating
\[
E_{t}[V_{t+1}(m_{t+1},p_{t+1},f_{t+1})]=\int_{\psi_{t}}\int_{\xi_{t}}\int_{\epsilon_{t}}V_{t+1}(m_{t+1},p_{t+1},f_{t+1})d\epsilon_{t}d\xi_{t}d\psi_{t}
\]
which can be done via Gauss-Hermite integration as all shocks are
log-normally distributed. We draw $8$ nodes for each of the three
stochastic elements $\{\psi_{j},\xi_{j},\epsilon_{j}\}_{j=1}^{8}$,
and we then calculate the expectation of the value function given
a realization of the state space $s_{t+1}=(m_{t+1},p_{t+1},f_{t+1})$
and a choice $(c_{t},\kappa_{t},i_{t})$:
\begin{gather*}
E_{t}[V_{t+1}(s_{t+1})]=E_{t}[\tilde{R}_{\kappa}(f_{t+1})a_{t}+Y_{t+1}+tr_{t+1}] =\\E_{t}\left[\tilde{R}_{\kappa}(f_{t+1})a_{t}+\left(\xi_{t+1}\left[Gp_{y,t}\psi_{t+1}\right]+g_{y,e}(t+1)\right)+tr_{t+1}\right]=\\
E_{t}\left[\left\{ (1-\kappa_{t})\bar{R}+\kappa_{t}\tilde{R}((1-\delta)f_{t}+i_{t})\right\} a_{t}+\left(\xi_{t+1}\left[Gp_{y,t}\psi_{t+1}\right]+g_{y,e}(t+1)\right)+tr_{t+1}\right]=\\
E_{t}\left[\left\{ (1-\kappa_{t})\bar{R}+\kappa_{t}\left[\bar{r}+r\left((1-\delta)f_{t}+i_{t}\right)+\epsilon_{t}\right]\right\} a_{t} + \left(\xi_{t+1}\left[Gp_{y,t}\psi_{t+1}\right]+g_{y,e}(t+1)\right)+tr_{t+1}\right]
\end{gather*}

which we can approximate by
\begin{gather*}
\sum_{j=1}^{8}w_{j}^{\psi}\sum_{i=1}^{8}w_{i}^{\xi}\sum_{k=1}^{8}w_{k}^{\epsilon} \\
\left\{ (1-\kappa_{t})\bar{R}+\kappa_{t}\left[\bar{r}+r\left((1-\delta)f_{t}+i_{t}\right)+\epsilon_{t+1}^{(k)}\right]\right\} \times \\ \left[m_{t}+tr_{t}-c_{t}-\pi(i_{t})-c_{d}I(\kappa_{t}>0)\right]+\left(\xi_{t+1}^{(i)}\left[Gp_{y,t}\psi_{t+1}^{(j)}\right]+g_{y,e}(t+1)\right)
\end{gather*}

\subsection{Notes}
\begin{itemize}
\item One interesting reformulation to the Danish context would be to make
pension tranfers fixed after age $65$ as a function of the wealth
at retirement.
\item In their formulation, $\tilde{R}$ is only log-normally distributed
meaning that we cannot obtain negative returns on our risky investments.
Perhaps that's why the model predicts high participation, whereas
this is not the case in real life?
\end{itemize}
